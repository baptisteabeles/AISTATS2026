
% \section{Mixing Bandit: main results}
\section{Linear bandits with non-i.i.d.~observation noise} \label{sec:contrib} 
We study a variant of the standard linear stochastic bandit problem where the observation-noise variables feature 
dependencies across different rounds. We focus on the case of weakly stationary noise, meaning we assume all the $\ee_t$ to have the same marginal distribution. %\eug{To check later: is this 
%actually used?} \grg{I think we only use a weaker form that is sometimes called ``weak stationarity'': that 
%the marginal expectation of the noise is zero at any $t$.}
However, the core assumption we make is what we call \emph{mixing sub-Gaussianity}. This provides a way to control how dependencies decay as the time between two 
observations increases. It is defined in terms of a sequence of mixing coefficients $\phi_d$, which quantify this decay. \begin{assumption}[Mixing sub-Gaussianity]
\label{ass:mixing-subgaussianity}
Fix $\sigma>0$ and let $\phi = (\phi_d)_{d\geq 0}$ be a non-negative and non-increasing sequence. We say that the random sequence $(\epsilon_t)_{t\geq 1}$ is \emph{$(\sigma, \phi)$-mixing sub-Gaussian} if $\ee_t$ is centred and $\sigma$-sub-Gaussian for every $t$, and, for all $d\geq 0$ and all $t>d$, we have
\begin{equation}
\label{eq::mixing}
    \bigl|\EEc{\epsilon_t}{\F_{t-d}} \bigr| \leq \phi_d
\end{equation}
and
\begin{equation}
\label{eq::mixing-subgaussianity}
   \EEc{\exp{\lambda(\epsilon_t-\EEc{\epsilon_t}{\F_{t-d}})}}{\F_{t-d}}\leq e^{\frac{\lambda^2\sigma^2}{2}}\,,
 \forall \lambda>0\,.
\end{equation}
\end{assumption}

Clearly, the above assumption generalises the standard conditionally sub-Gaussian assumption (that can be recovered by 
setting $\phi_d=0$ for all $t$), sometimes considered in the bandit literature. Although this might look like an 
unusual mixing assumption, it is very natural for our problem at hand, and can be weaker than standard mixing 
hypotheses. For instance, if the noise sequence is $\varphi$-mixing (see \citealp{bradley2005basic}) and each $\ee_t$ 
is centred and bounded in $[-a,b]$, it is straightforward to check that $|\E[\ee_t|\F_{t-d}]| \leq (a+b)\phi_d$, and so 
Assumption \ref{ass:mixing-subgaussianity} is satisfied since the boundedness automatically implies sub-Gaussianity. In the rest of the paper we assume $\sigma =1$ for simplicity.

Under Assumption \ref{ass:mixing-subgaussianity}, we can build the confidence sequence needed for our UCB algorithm. We 
state this result below, but defer the explicit derivation to \Cref{sec:conf_sets} (see Corollary \ref{cor:conf} there). %\eug{The statement below needs to be checked once derivation in sec 4 is completed...}

\begin{prop}%[Confidence sequence under mixing sub-Gaussianity]
 \label{prop:CS_Mixing_Bandit} For some given $\phi$, let the noise satisfy Assumption \ref{ass:mixing-subgaussianity} with $\sigma=1$. Fix $\delta\in(0,1)$, $\lambda>0$, and $d\geq 1$. For $t\geq 1$ let \begin{align*}
 \mathcal\C_t = \Big\{\theta\in\mathcal B(B)\,:\,\tfrac{1}{2}\|\theta-\wh\theta_t\|_{V_t}^2&\leq \tfrac{dp}{2}\log\tfrac{(B+1)^2e\max(dp,t+d)}{dp} 
\\&+ 2\lambda B^2 + t\phi_d (B+1) + d\log\tfrac{d}{\delta}\Big\}\,,
 \end{align*} where $V_t = \sum_{s=1}^t X_tX_t^\top + \lambda\mathrm{Id}$, and $\wh\theta_t = \argmin_{\theta\in\mathcal B(B)} \sum_{s=1}^t(\langle \theta,X_t\rangle -Y_t)^2$. Then, $(\C_t)_{t\geq 1}$ is an anytime valid confidence sequence, in the sense that
 $$\mathbb P\big(\theta^\star\in \C_t\,,\;\forall t\geq 1\big)\geq 1-\delta\,.$$
 \end{prop}
 
 %\begin{corollary}(Confidence Sets for Geometric Mixing)
 %\label{coro:geometrix_mixing}
 %    Assume $(\epsilon_t)_{t=1}^n$ satisfies Assumption \ref{ass:mixing-subgaussianity} with $\phi_d = Ce^{-\frac{d}{\tau}}$ for some $C,\tau>0$
 %    \[\mathbb{P}\left[\forall t >0, \theta^* \in C_t(\delta)\right] \geq 1-\delta,\]
 %    where we define
 %    \begin{equation}
 %    \label{eq:theorem1}C_t(\delta):= \left\{\theta, \LL_t(\theta)-\LL_t(\wh \theta_t) \leq dp\log(t/dp) + BCte^{-\frac{d}{\tau}} + d \log\frac{d}{\delta}\right\} 
 %    \end{equation}
 %\end{corollary}
 %\begin{corollary}(Confidence Sets for Algebraic Mixing)
 %\label{coro:algebraic_mixing}
 %     Assume $(\epsilon_t)_{t=1}^n$ satisfies Assumption \ref{ass:mixing-subgaussianity} with $\phi_d = Cd^{-r}$ for some $C,r>0$
 %    \[\mathbb{P}\left[\forall t >0, \theta^* \in C_t(\delta)\right] \geq 1-\delta,\]
 %    where we define
 %    \begin{equation}
 %    \label{eq:theorem1}C_t(\delta):= \left\{\theta, \LL_t(\theta)-\LL_t(\wh \theta_t) \leq dp\log(t/dp) + BCtd^{-r} + d \log\frac{d}{\delta}\right\} 
 %    \end{equation}
 %\end{corollary}

Leveraging the confidence sequence above, we can define a UCB approach for our problem (\Cref{alg::Mixing_LinUCB}). At a high level, the algorithm 
 operates by taking the confidence sets defined in Proposition 
\ref{prop:CS_Mixing_Bandit}, and selecting the arm optimistically, as in the standard UCB. A key point is that a delay 
$d$ is introduced, which at round $t$ restricts the agent to use only the information available from the first $t-d$ 
rounds. Although the actual technical reason behind this restriction will become fully clear only with the analysis of the 
coming sections, one can intuitively think of it as a way to prevent overfitting to recent noise, which might be 
highly correlated. If $d$ is sufficiently large, the noise observed in each round $t$ will 
be sufficiently decorrelated from the previous observations, which allows accurate estimation and uncertainty 
quantification of the true parameter $\theta^\star$ and the associated rewards.  

\begin{algorithm}[!h]
\caption{Mixing-LinUCB}\label{alg::Mixing_LinUCB}
\begin{algorithmic}
%(\frac{1}{K},...,\frac{1}{K})
\State set $d>0$
\For {$i \in \{1,2,\dots d\}$}
\State play an arbitrary $X_i$ and observe $Y_i$
\EndFor
\For {$t \in \{d+1,\dots\}$}
\State  $X_{t} = \arg\max_{x \in \X_t} \ucb_{\C_{t-d}}(x)$, where $\C_{t-d}$ is as in Proposition \ref{prop:CS_Mixing_Bandit}
\State play $X_t$ and observe reward $Y_t$
\EndFor
\end{algorithmic}
\end{algorithm} 

In Section \ref{sec:bandit_regret} we provide a detailed analysis of the regret of the algorithm that we proposed. For instance, assuming that the mixing coefficients decay exponentially as $\phi_d = C e^{-d/\tau}$ (\emph{geometric mixing}), we show that the regret can be upper bounded in high probability as  \[\Reg(T) \leq \OO\left(\tau p\sqrt{T}\log(T)^2+\tau \log T\sqrt{pT \log T}\right).\]
We refer to Theorem \ref{thm:worst_case_reg} and Corollary \ref{cor:geometric_mixing} in Section \ref{sec:bandit_regret} for more details.