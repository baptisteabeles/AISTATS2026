\section{Introduction}
The linear bandit problem \citep{abe1999associative, auer2003using} is an instance of a multi-armed bandit framework, 
where the expected reward is linear in the feature vector representing the chosen arm. More concretely, it is a 
sequential decision-making problem, where an agent each round picks an arm $X_t$, and receives a reward $Y_t = 
\siprod{\theta^\star}{X_t} + \ee_t$, with $\theta^\star$ a fixed parameter unknown to the agent, and $\ee_t$ zero-mean 
random noise. This framework has gained significant attention in the literature as it yields analytic tools that can be 
applied to several concrete applications, such as online advertising \citep{abe2003reinforcement}, recommendation 
systems \citep{li2010contextual, korkut2021disposable}, and dynamic pricing \citep{cohen2020feature}. 

A popular strategy to tackle linear bandits leverages the principle of \emph{optimism in the face of uncertainty}, via 
upper confidence bound (UCB) algorithms. The idea of optimism can be traced back to \citet{lai1985asymptotically}, and 
its application to linear bandits was already advanced by \citet{auer2003using}. Since then, this approach has been 
improved and analysed by several works \citep{abbasi2011improved, lattimore2020bandit, flynn2023improved}. This class of 
methods requires constructing an adaptive sequence of confidence sets that, with high probability, contain the true 
parameter $\theta^\star$. Each round, the agent selects the arm maximising the expected reward under the most optimistic 
parameter (in terms of reward) in the current confidence set. UCB-based algorithms have become popular as they are often 
easy to implement and come with tight worst-case regret guarantees. 

For a UCB algorithm to perform well, it is necessary that the confidence sets are tight, which can be ensured by taking 
advantage of the structure of the problem. In this paper, our focus is on studying various assumptions on the 
observation noise. A commonly studied situation is when $(\ee_t)_{t\geq 0}$ consists of a sequence of 
i.i.d.~realisations of some bounded or sub-Gaussian random variable (see \citealp{lattimore2020bandit}, Chapter 20). 
Often, the standard analysis can be extended to the case in which the realisation are not independent, but conditionally 
centred and sub-Gaussian \citep{abbasi2011improved}. Yet, in real-world settings, this assumption is often unrealistic, 
as one can expect the presence of interdependencies among the noise at different rounds. For instance, in the context of 
advertisement selection, the noise models the ensemble of external factors that influence the user's choice on whether 
to click or not an ad. The i.i.d.~assumption implies that across different rounds these external factors are completely 
independent. In practice, the user choice will be affected by temporally correlated events, such as recent browsing 
history or exposure to similar content. Therefore, a more realistic assumption is to allow the dependencies to decay 
with time, rather than being completely absent. This way to model dependencies, often referred to as \emph{mixing}, is common to study concentration for sums of non-i.i.d.~random variables, with applications to machine learning \citep{bradley2005basic, mohri2008rademacher,abeles2024generalization}.

In the present paper we relax the assumption that the noise is conditionally zero-mean in the bandit problem, and we allow for the presence of dependencies. Concretely, we replace  the standard 
conditionally sub-Gaussian setting with a more general formulation that accounts for conditional 
dependence of the noise on the past, by introducing a natural notion of \emph{mixing sub-Gaussianity}. Within this 
context, we introduce a UCB algorithm for which we rigorously establish regret guarantees. There are two key challenges 
for our approach: constructing a valid confidence sequence under dependent noise, and deriving a regret upper bound for 
the UCB algorithm that we propose. 

We derive the confidence sequence by adapting the \emph{online-to-confidence-sets} technique to accommodate temporal 
dependencies in the noise. This approach, originally introduced by \cite{abbasi2011improved} and recently extended and 
improved \citep{jun2017scalable, lee2024improved, clerico2025confidence}, involves 
constructing an abstract online learning game whose regret guarantees can be turned into a confidence sequence. To deal 
with the dependencies in the noise, we modify the standard online-to-confidence-sets framework by introducing delays in 
the feedback received within the abstract online game. This approach is inspired by the recent work of 
\cite{abeles2024generalization} on extending online-to-PAC conversions to non-i.i.d.~mixing data sets in the context of 
deriving generalisation bounds for statistical learning. There, a  delayed-feedback trick similar to ours is employed to 
derive statistical guarantees (generalisation bounds) from an abstract online learning game. 

For the regret analysis of the bandit algorithm, we also need to face some challenges due to the correlated 
observation noise. We address these by introducing delays into the decision-making policy as well.
% adapting the classic 
% ``elliptical potential lemma'' (cf.~Lemma 19.4 in \citealp{lattimore2020bandit}) to our setting, in order to 
% upper-bound the width of the confidence sets.
%\bapt{To be honest I really not see how to connect delayed bandit with our setting here because it seems in delayed 
% bandit they still get a UCB algorithm without delay}
This makes our approach superficially similar to algorithms used in the rich literature on bandits with delayed feedback
(see, e.g., \citealp{vernade2020linear,howson2023delayed}). These works consider delay as part of the problem statement 
and not part of the solution concept, and are thus orthogonal to our work. In particular, a simple adaptation of 
results from this literature would not suffice for dealing with dependent observations, which we tackle by developing 
new concentration inequalities. Another line of work that is conceptually related to ours is that of non-stationary 
bandits \citep{garivier2008upper,russac2019weighted}. In that setting, the parameter vector $\theta^\star_t$ evolves in 
time according to a nonstationary stochastic process, and the observation noise remains i.i.d., once again making for a 
rather different problem with its own challenges. Namely, the main obstacle to overcome is that comparing with the 
optimal sequence of actions becomes impossible unless strong assumptions are made about the sequence of 
parameter vectors. A typical trick to deal with these nonstationarities is to discard old observations (which may have 
been generated by a very different reward function), and use only recent rewards for decision-making. This is the polar 
opposite of our approach that is explicitly \emph{disallowed} to use recent rewards, which clearly highlights how 
different these problems are. That said, there exists an intersection between the worlds of delayed and nontationary 
bandits \citep{vernade2020non}, and thus we would not discard the possibility of eventually building a bridge between 
bandits with nonstationary reward functions and bandits with nonstationary observation noise. For simplicity, we focus 
on the second of these two components in this paper.

% 
% The latter assumes that the unknown parameter is time-varying, \emph{i.e.} $Y_t = \siprod{\theta_t}{X_t}$. To compare it with the classical formulation assume $(\theta_t)_{t>0}$ is drawn from a stationary process and denote $\theta^\star= \EE{\theta_t}$ then we can write $Y_t = \siprod{\theta^\star}{X_t} + \ee_t$ where $\ee_t = \siprod{\theta_t-\theta^\star}{X_t}.$ This simple remark suggests that the methods developed in this work could be interesting for non stationary bandit, as our assumptions on $\ee_t$ could possibly be considered to model the variations of the unknown $\theta_t.$ 







 %(\citealp{abbasi2011improved},\citealp{flynn2023improved}, \cite{lattimore2020bandit}, \cite{orabona2019modern}, \cite{lee2024unified} \cite{clerico2025confidencesequencesgeneralizedlinear}
 
 \paragraph{Notation.}
 Throughout the paper, we will often use the following notations. For $u$ and $v$ in $\real^p$, we let $\langle u, v\rangle$ denote their dot product. $\|u\|_2 = \sqrt{\langle u, u\rangle}$ is the Euclidean norm, while for a non-negative definite $(p\times p)$-matrix $A$, $\|u\|_A = \sqrt{\langle u, Au\rangle}$ is a semi-norm (a norm if the matrix is strictly positive definite). For $r>0$, $\mathcal B(r)$ denotes the closed centred Euclidean ball in $\real^p$ with radius $r$. Given a non-empty set $U\subseteq\real^p$, we let $\Delta_U$ denote the space of (Borel) probability measures on $\real^p$ whose support in $U$. Finally, $(u_t)_{t\geq t_0}$ denotes a sequence indexed on the integers, with $t_0$ its smallest index. 
